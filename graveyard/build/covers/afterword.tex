\documentclass[12pt]{article}
\usepackage{fontspec}
\usepackage[utf8]{inputenc}
\setmainfont{Bodoni 72 Book}
\usepackage[paperwidth=9in,paperheight=12in,margin=1in,headheight=0.0in,footskip=0.5in,includehead,includefoot,portrait]{geometry}
\usepackage[absolute]{textpos}
\TPGrid[0.5in, 0.25in]{23}{24}
\parindent=0pt
\parskip=12pt
\usepackage{nopageno}
\usepackage{graphicx}
\graphicspath{ {./images/} }
\usepackage{amsmath}
\usepackage{tikz}
\newcommand*\circled[1]{\tikz[baseline=(char.base)]{
            \node[shape=circle,draw,inner sep=1pt] (char) {#1};}}

\begin{document}

\vspace*{3\baselineskip}

\begin{center}
\huge AFTERWORD
\end{center}

\vspace*{2\baselineskip}

\begingroup
\hspace{10mm} The use of \textbf{graveyard dirt} in ritual magic has its roots in the traditions of the Bakongo people of Central Africa, who believed that earth from a grave contained the spirit of the person buried in it. This belief was brought to the Southeastern United States by enslaved people from the Kongo Empire starting \(\sim \) the 1730's, afterwards influencing the magical traditions of many cultures within the region. This includes but is not limited to the Conjure tradition of Georgia, the Hoodoo traditions of Louisiana and Mississippi, several Appalachian folk magic practices, and the many varieties of contemporary magical religions practiced throughout the American Southeast today.
\endgroup

\begingroup
\hspace{10mm} Graveyard dirt must be purchased from the spirit occupying the relevant grave. This entails contacting the spirit, a respectful request to use the dirt, and the exchange of payment: often money, such as the traditional silver dime left by Hoodoo practitioners, or alcohol. In the selection of the grave, the temperament of the spirit is considered. To bring good fortune, a baby's spirit may be used. For love, dirt from above the heart of a buried loved one. For a curse, the spirit of a murderer or other malevolent figure.
\endgroup

\begingroup
\hspace{10mm} Those who used graveyard dirt or practiced the traditions it was associated with were considered troublemakers, deviants, and oftentimes aligned against the orthodox Christian beliefs which made up the religious majority of the region. This lead to several false accusations of evildoing, especially against members of racial, religious, and socioeconomic minority groups. These accusations were often accompanied by physical punishment, refusal of housing, ostracization, and expulsion from work, regardless of the accuracy of the accusation. It seems that the easiest way to acquire graveyard dirt, was to be accused of its usage.
\endgroup

\end{document}