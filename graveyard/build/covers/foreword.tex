\documentclass[12pt]{article}
\usepackage{fontspec}
\usepackage[utf8]{inputenc}
\setmainfont{Bodoni 72 Book}
\usepackage[paperwidth=9in,paperheight=12in,margin=1in,headheight=0.0in,footskip=0.5in,includehead,includefoot,portrait]{geometry}
\usepackage[absolute]{textpos}
\TPGrid[0.5in, 0.25in]{23}{24}
\parindent=0pt
\parskip=12pt
\usepackage{nopageno}
\usepackage{graphicx}
\graphicspath{ {./images/} }
\usepackage{amsmath}
\usepackage{hyperref}
\usepackage{tikz}
\newcommand*\circled[1]{\tikz[baseline=(char.base)]{
            \node[shape=circle,draw,inner sep=1pt] (char) {#1};}}

\begin{document}

\begin{center}
\huge FOREWORD
\end{center}

\vspace*{1\baselineskip}

\begingroup
\begin{center}
\leftskip0.5in
``Graveyard dirt must be got off the coffin of the dead person, on the waste of the moon at midnight . . . 
\end{center}
\endgroup

\begingroup
\begin{center}
\leftskip0.5in
``To Conjure a well, throw into the well graveyard dirt, an old pipe of a Conjure Doctor, or some devil's snuff . . . 
\end{center}
\endgroup

\begingroup
\begin{center}
\leftskip0.5in
``Conjure as graveyard dirt is taken from a grave one day after burial . . . 
\end{center}
\endgroup

\begingroup
\begin{center}
\leftskip0.5in
``One can be Conjured by shaking hands with any one, if he has rubbed his hands with graveyard dirt . . . 
\end{center}
\endgroup

\begingroup
\begin{center}
\leftskip0.5in
``To sprinkle graveyard dirt about the yard, about a house, makes one sleepy, sluggish, naturally waste away and perish until he dies . . ." 
\rightskip\leftskip
\phantom{text} \hfill - Roland Steiner ( \textit{from various informants} )
\end{center}
\endgroup

\vspace*{1\baselineskip}

\begin{center}
\huge NOTES FOR THE INTERPRETERS
\end{center}

\vspace*{1\baselineskip}

\begingroup
\textbf{General: \circled{1} Dynamics} in the \textbf{viola} and \textbf{accordion} correspond to \textbf{implement pressure} ( such as the fingers, the bellows, or the bow ). Dynamics in the \textbf{electric guitar} correspond to the \textbf{volume as controlled by a foot pedal}. \textbf{\circled{2}} After temporary \textbf{accidentals}, cancellation marks are printed also in the following measure ( for notes in the same octave ) and, in the same measure, for notes in other octaves, but they are printed again if the same note appears later in the same measure, except if the note is immediately repeated. \textbf{\circled{3}} Each interpreter reads \textbf{two staves}, where the \textbf{upper staff} corresponds to the actions of the \textbf{right hand}, and the \textbf{lower staff} corresponds to the actions of the \textbf{left hand}. If only one staff is present, the division of the hands is left to the interpreter's discretion. \textbf{\circled{4} Playing techniques} apply only to the note to which they are attached. If a technique is to persist for longer than a single note, a hooked, dashed line will span the music as long as the technique is active. \textbf{\circled{5} Dashed arrows above the staff} indicate a gradual transition from one technique or tempo to another. \textbf{\circled{6} Time signatures whose denominators are not a power of two} are to be understood as a type of metric modulation wherein the pulse shifts to a prolation indicated by the denominator. For example, \textbf{1/6} will contain one ``sixth" note, which is one-sixth of a whole note, or, a triplet quarter note. When these time signatures are active, tuplet brackets which are open on the right side similarly indicate the prolation of a note alone, rather than the number of beats in the prolation. \textbf{\circled{7} Blank measures} are to be understood as full-measure rests. \textbf{\circled{8} Flat glissandi} are sometimes used for the same function as ties.
\endgroup

\pagebreak

\begingroup
\textbf{Scordatura: \circled{1} The electric guitar}'s \textbf{sixth string} should be tuned down a minor third to \textbf{C - sharp 3}. The \textbf{fifth string} should be tuned up a major third to \textbf{C - sharp 4}. The \textbf{fourth string} should be tuned up $\sim$ a major second to \textbf{E 4} at a ratio of \textbf{7/6} of the fifth string ( or, a \textbf{septimal minor third} above the fifth string ). The \textbf{third string} should be tuned at a ratio of \textbf{6/5} of the fourth string ( or, a \textbf{just minor third} above the fourth string ). The \textbf{second string} should be tuned at a ratio of \textbf{14/11} of the third string ( or, an \textbf{undecimal major third} above the third string ). This will sound $\sim$ as a \textbf{B 4}. The \textbf{first string} should be tuned up $\sim$ a minor second to \textbf{F 5} at a ratio of \textbf{11/8} of the second string ( or, the \textbf{11th partial} of the second string lowered one octave ). \\ \textbf{\circled{2}} The tuning of the open strings represented with Helmholtz-Ellis accidentals coupled with the deviation in cents from the nearest “standard” accidental is below:

\begin{center}
\includegraphics[scale=0.48]{guitar_scord.png}
\end{center}

\textbf{\circled{3} The viola} should be tuned with each string at a ratio of \textbf{3/2} of the string beneath it ( or, each string should be tuned to a \textbf{pure fifth} above the string beneath it ). \\ \textbf{\circled{4}} The tuning of the open strings represented with the deviation in cents from the nearest equally tempered pitch is below:

\begin{center}
\includegraphics[scale=0.38]{viola_scord.png}
\end{center}

\textbf{\circled{5} For aid in audiation}, a Supercollider file which plays the exact pitches of the open strings can be found at this link:
\endgroup

\begingroup
\begin{center}
\url{https://github.com/tr1nt0n/graveyard/blob/main/graveyard/etc/microtones/voice_to_sc_2022-11-20-21-17-26-750852.scd} , 
\end{center}
\endgroup

\begingroup
with the \textbf{guitar's pitches} enclosed in the \textbf{Pbind} starting at \textbf{line 40}, and the \textbf{viola's pitches} enclosed in the \textbf{Pbind} starting at \textbf{line 94}. \\
\textbf{\circled{6} This score is transposed} to the physical playing position on the string rather than sounding pitch.
\endgroup

\end{document}