\documentclass[12pt]{article}
\usepackage{fontspec}
\usepackage[utf8]{inputenc}
\setmainfont{Bodoni 72 Book}
\usepackage[paperwidth=9in,paperheight=12in,margin=1in,headheight=0.0in,footskip=0.5in,includehead,includefoot,portrait]{geometry}
\usepackage[absolute]{textpos}
\TPGrid[0.5in, 0.25in]{23}{24}
\parindent=0pt
\parskip=12pt
\usepackage{nopageno}
\usepackage{graphicx}
\graphicspath{ {./images/} }
\usepackage{amsmath}
\usepackage{hyperref}
\usepackage{tikz}
\newcommand*\circled[1]{\tikz[baseline=(char.base)]{
            \node[shape=circle,draw,inner sep=1pt] (char) {#1};}}

\begin{document}

\begin{center}
\huge FOREWORD
\end{center}

\begingroup
\begin{center}
\leftskip0.5in
``Graveyard dirt must be got off the coffin of the dead person, on the waste of the moon at midnight . . . 
\end{center}
\endgroup

\begingroup
\begin{center}
\leftskip0.5in
``To Conjure a well, throw into the well graveyard dirt, an old pipe of a Conjure Doctor, or some devil's snuff . . . 
\end{center}
\endgroup

\begingroup
\begin{center}
\leftskip0.5in
``Conjure as graveyard dirt is taken from a grave one day after burial . . . 
\end{center}
\endgroup

\begingroup
\begin{center}
\leftskip0.5in
``One can be Conjured by shaking hands with any one, if he has rubbed his hands with graveyard dirt . . . 
\end{center}
\endgroup

\begingroup
\begin{center}
\leftskip0.5in
``To sprinkle graveyard dirt about the yard, about a house, makes one sleepy, sluggish, naturally waste away and perish until he dies . . ." 
\rightskip\leftskip
\phantom{text} \hfill - Roland Steiner ( \textit{from various informants} )
\end{center}
\endgroup

\begin{center}
\huge NOTES FOR THE INTERPRETERS
\end{center}

\begingroup
\textbf{General: \circled{1} Dynamics} in the \textbf{viola} and \textbf{accordion} correspond to \textbf{implement pressure} ( such as the fingers, the bellows, or the bow ). Dynamics in the \textbf{electric guitar} correspond to the \textbf{volume as controlled by a foot pedal}. \textbf{\circled{2}} After temporary \textbf{accidentals}, cancellation marks are printed also in the following measure ( for notes in the same octave ) and, in the same measure, for notes in other octaves, but they are printed again if the same note appears later in the same measure, except if the note is immediately repeated. \textbf{\circled{3} The equally tempered accidentals} used in this score are \textbf{semi tones}, \textbf{quarter tones}, and \textbf{cents}. A \textbf{quarter tone flat} is represented by an \textbf{inverted flat symbol}, and a \textbf{quarter tone sharp} indicated by a \textbf{sharp symbol with only one vertical line}. \textbf{Cents} are indicated with a \textbf{plus} or \textbf{minus} sign followed by the \textbf{number of cents} to be added to or subtracted from the \textbf{12 - tone equally tempered} pitch. \textbf{\circled{4}} Each interpreter reads \textbf{two staves}, where the \textbf{upper staff} corresponds to the actions of the \textbf{right hand}, and the \textbf{lower staff} corresponds to the actions of the \textbf{left hand}. If only one staff is present, the division of the hands is left to the interpreter's discretion. \textbf{\circled{5} Playing techniques} apply only to the note to which they are attached. If a technique is to persist for longer than a single note, a hooked, dashed line will span the music as long as the technique is active. \textbf{\circled{6} Dashed arrows above the staff} indicate a gradual transition from one technique or tempo to another. \textbf{\circled{7} Time signatures whose denominators are not a power of two} are to be understood as a type of metric modulation wherein the pulse shifts to a prolation indicated by the denominator. For example, \textbf{1/6} will contain one ``sixth" note, which is one-sixth of a whole note, or, a triplet quarter note. When these time signatures are active, tuplet brackets which are open on the right side similarly indicate the prolation of a note alone, rather than the number of beats in the prolation. \textbf{\circled{8} Tremoli} are not to be interpreted as precise subdivisions of a note, but their \textbf{approximate speed} is determined by \textbf{number of slashes}, wherein \textbf{one} slash indicates \textbf{tremolo largo}, \textbf{two} indicates \textbf{tremolo moderato}, and \textbf{three} indicates \textbf{tremolo stretto}. Note that tremolo speeds may be interpolated, signified by tremolo slashes above the staff connected by a dashed arrow. \textbf{\circled{9} Blank measures} are to be understood as full-measure rests. \textbf{\circled{10} Flat glissandi} are sometimes used for the same function as ties. \textbf{\circled{11} Sections} delineated by double bar lines and rehearsal marks are to be understood as separate movements, but should be played attacca, especially maintaining the dynamic transitions between the movements.

\endgroup

\pagebreak

\begingroup
\textbf{Staging: \circled{1} The auxiliary percussion instruments} used in this piece are \textbf{one large bass drum}, shared by the ensemble, and \textbf{three single zhongbo} ( {\setmainfont{Source Han Serif SC}\selectfont\textbf{中钹}} ), each mounted. \textbf{\circled{2} The common implements} which \textbf{all interpreters} should have at their disposal are \textbf{a drumstick} ( for the bass drum ), and \textbf{a bow} ( for their respective zhongbo ). \textbf{\circled{3}} The interpreters should be sat in a \textbf{triangle} formation around the \textbf{upturned} bass drum, \textbf{each facing the drum}, with the \textbf{guitar} at \textbf{stage left}, the \textbf{viola} at \textbf{center}, and the \textbf{accordion} at \textbf{stage right}, with each of their zhongbo to their \textbf{right}. 
\endgroup

\begingroup
\textbf{Electric Guitar: \circled{1} This score is transposed } so that the written pitch is \textbf{one octave} above the sounding pitch. \textbf{\circled{2} The timbre} of the guitar should be distorted enough that there is noticeable string noise caused by the left hand ( especially when leaping between large intervals ), without \textit{completely} obscuring the harmony. \textbf{\circled{3} Pedals} are not changed for the duration of the piece, with the exception of a volume pedal, which should be used to interpret all dynamic markings. \textbf{\circled{4}} Despite scordatura, \textbf{stringing} is largely left to the discretion of the interpreter to introduce an aleatoric layer to the harmony, except for when a \textbf{six - line staff} is present in the \textbf{right hand}, wherein the \textbf{top line} indicates to play on \textbf{string I}, the next line on string II, and so on. \textbf{\circled{5} Spazzolato} indicates to drag the pick vertically across the strings. This technique is always accompanied by a \textbf{two - line staff} in the right hand, wherein the \textbf{top line} indicates \textbf{just before the bridge}, the \textbf{bottom line} indicates \textbf{halfway up the fingerboard}, and the \textbf{space between the lines} indicates \textbf{approximate positions} between the two. \textbf{\circled{6} When playing with the vibrato bar}, the interpreter reads the \textbf{same two - line staff}, wherein the \textbf{top line} indicates an \textbf{uncompressed bar}, the bottom line indicates a bar pressed \textbf{down as far as possible}, and the \textbf{space between the lines} indicates \textbf{approximate positions} between the two. \textbf{\circled{7} The two types of rasgueado} in this score are \textbf{knuckle rasgueado} and \textbf{nail rasgueado.} The second of these, \textbf{nail rasgueado}, is the familiar four - finger strumming technique originating in flamenco. \textbf{Knuckle rasgueado} is a modification of this technique played with the knuckles instead of fingernails. Note that the score requests gradual transitions of color between these two types of rasgueado. \textbf{\circled{8} The abbreviations} used in this score are \textbf{pont.} for \textbf{sul ponticello}, \textbf{tast.} for \textbf{sul tasto}, \textbf{scratch} for \textbf{scratch tone}, \textbf{vib.} for \textbf{vibrato bar}, \textbf{tap} for \textbf{finger tapping}, \textbf{kn. rasg.} for \textbf{knuckle rasgueado}, and \textbf{n. rasg.} for \textbf{nail rasgueado}.
\endgroup

\begingroup
\textbf{Viola: \circled{1} The viola should be amplified} to balance with the electric guitar, but especially to clarify the sounds of a \textbf{ring of keys or other light metals} on the strings and \textbf{bowing on the body of the instrument}. \textbf{\circled{2} Key rattling} indicates to dangle a ring of keys or other light metals in contact with the strings ( strings are specified via the four-line staff described under the next point ), and gently shake the ring such that it produces a tremolo. \textbf{\circled{3} A four - line staff} indicates to play on the \textbf{open strings}, wherein the \textbf{top line} indicates to play on \textbf{string I}, the next line on string II, and so on. \textbf{\circled{4} Spazzolato} indicates to drag the bow vertically across the strings. This technique is always accompanied by a \textbf{two - line staff} in the right hand, wherein the \textbf{top line} indicates \textbf{just before the bridge}, the \textbf{bottom line} indicates \textbf{halfway up the fingerboard}, and the \textbf{space between the lines} indicates \textbf{approximate positions} between the two. Note that this staff may also be used in other situations where ponticello - tasto transitions are needed. \textbf{\circled{5}} In various passages throughout this piece, there is notation which represents \textbf{the point at which the bow is touched} as it is drawn across the string. These positions are written as \textbf{fractions} where \textbf{0/7} and \textbf{0/5} represent \textbf{au talon} and \textbf{7/7} and \textbf{5/5} represent \textbf{punta d`arco}. For the duration of the note to which these fractions are attached, the interpreter should draw the bow at a constant speed, moving toward the destination point indicated on the following note. Bowings are provided. Passages without these indications should be bowed at the interpreter’s discretion. \textbf{\circled{6} Cross - shaped note heads} indicate to \textbf{damp the string}, removing as much pitch from the sound as possible. \textbf{\circled{7} The abbreviations} used in this score are \textbf{pont.} for \textbf{sul ponticello}, \textbf{tast.} for \textbf{sul tasto}, \textbf{scratch} for \textbf{scratch tone}, \textbf{legno bat.} for \textbf{col legno battuto}, \textbf{vib.} for \textbf{vibrato}, \textbf{ring} for \textbf{key rattling}, and \textbf{body} for \textbf{bowing on the body}.
\endgroup

\pagebreak

\begingroup
\textbf{Accordion: \circled{1} The accordion should be amplified} to balance with the electric guitar, but especially to clarify the sounds of \textbf{key clicking}, \textbf{flapping the bellows}, and \textbf{air sound}. \textbf{\circled{2} When clicking the keys}, signified by a \textbf{cross - shaped note head} on a \textbf{single - line staff}, the decision of which keys to click is left to the interpreter. However, key clicks in the \textbf{top staff} correspond to the \textbf{keyboard keys}, and key clicks in the \textbf{bottom staff} correspond to the \textbf{button keys}. \textbf{\circled{3} When flapping the bellows}, the interpreter should place their hand between the center - most bellows, and move the hand to the left or right, producing sound by the strikes of the bellows. Note that a \textbf{single note} corresponds to a \textbf{single strike}, and should not be performed as tremolo unless specified with a \textbf{stem tremolo}. \textbf{\circled{4} The abbreviations} used in this score are \textbf{air} for \textbf{air button}, \textbf{key} for \textbf{key clicking}, \textbf{trem.} for \textbf{tremolo}, and \textbf{bellow} for \textbf{bellow flapping}.
\endgroup

\begingroup
\textbf{Scordatura: \circled{1} The electric guitar}'s \textbf{sixth string} should be tuned down a minor third to \textbf{C - sharp 2}. The \textbf{fifth string} should be tuned up a major third to \textbf{C - sharp 3}. The \textbf{fourth string} should be tuned up $\sim$ a major second to \textbf{E 3} at a ratio of \textbf{7/6} of the fifth string ( or, a \textbf{septimal minor third} above the fifth string ). The \textbf{third string} should be tuned at a ratio of \textbf{6/5} of the fourth string ( or, a \textbf{just minor third} above the fourth string ). The \textbf{second string} should be tuned at a ratio of \textbf{14/11} of the third string ( or, an \textbf{undecimal major third} above the third string ). This will sound $\sim$ as a \textbf{B 3}. The \textbf{first string} should be tuned up $\sim$ a minor second to \textbf{F 4} at a ratio of \textbf{11/8} of the second string ( or, the \textbf{11th partial} of the second string lowered three octaves ). \\ \textbf{\circled{2}} The tuning of the open strings represented with Helmholtz-Ellis accidentals coupled with the deviation in cents from the nearest “standard” accidental is below:

\begin{center}
\includegraphics[scale=0.48]{guitar_scord.png}
\end{center}

\textbf{\circled{3}} It is preferred but not necessary that \textbf{the viola} be tuned in \textbf{pure fifths} ( at a ratio of \textbf{3/2} ) as opposed to equal temperament. 

\textbf{\circled{4} For aid in audiation}, a Supercollider file which plays the exact pitches of the open strings can be found at this link:
\endgroup

\begingroup
\begin{center}
\url{https://github.com/tr1nt0n/graveyard/blob/main/graveyard/etc/microtones/voice_to_sc_2023-01-13-19-14-29-469899.scd} , 
\end{center}
\endgroup

\begingroup
with the \textbf{guitar's pitches} enclosed in the \textbf{Pbind} starting at \textbf{line 40}, and the \textbf{viola's pitches} enclosed in the \textbf{Pbind} starting at \textbf{line 94}. \\
\textbf{\circled{5} This score is transposed} to the physical playing position on the string rather than sounding pitch.
\endgroup

\end{document}